\documentclass[12pt]{article}
\usepackage{graphicx}
%\documentclass[journal,12pt,twocolumn]{IEEEtran}
\usepackage[none]{hyphenat}
\usepackage{graphicx}
\usepackage{listings}
\usepackage[english]{babel}
\usepackage{graphicx}
\usepackage{caption} 
\usepackage{hyperref}
\usepackage{booktabs}
\usepackage{commath}
\usepackage{gensymb}
\usepackage{array}
\usepackage{amsmath}   % for having text in math mode
\usepackage{listings}
\lstset{
  frame=single,
  breaklines=true
}
  
%Following 2 lines were added to remove the blank page at the beginning
\usepackage{atbegshi}% http://ctan.org/pkg/atbegshi
\AtBeginDocument{\AtBeginShipoutNext{\AtBeginShipoutDiscard}}
%


%New macro definitions
\newcommand{\mydet}[1]{\ensuremath{\begin{vmatrix}#1\end{vmatrix}}}
\providecommand{\brak}[1]{\ensuremath{\left(#1\right)}}
\providecommand{\abs}[1]{\left\vert#1\right\vert}
\providecommand{\norm}[1]{\left\lVert#1\right\rVert}
\newcommand{\solution}{\noindent \textbf{Solution: }}
\newcommand{\myvec}[1]{\ensuremath{\begin{pmatrix}#1\end{pmatrix}}}
\let\vec\mathbf


\begin{document}
\begin{center}
\title{\textbf{Properties of vectors}}
\date{\vspace{-5ex}} %Not to print date automatically
\maketitle
\end{center}
\setcounter{page}{1}
\section{12$^{th}$ Maths - Exercise 10.4.2}

\begin{enumerate}
\item Find a unit vector  perpendicular to each of a vector $\bar{a}+\bar{b} \text{ and }\bar{a}-\bar{b}$ where  $\overrightarrow{a}=3\hat{i}+2\hat{j}+2\hat{k}\text{ and }\overrightarrow{b}=\hat{i}+2\hat{j}-2\hat{k}$
\section{Solution}
Now,
\begin{align}
\text{Let } \vec{A} = \myvec{3\\2\\2} \text{ and } \vec{B} = \myvec{1\\ 2 \\ -2}
\end{align}
The cross product or vector product of $\vec{A},\vec{B}$ is defined as
\begin{align}
 \vec{A} \times \vec{B} = \myvec{\mydet{\vec{A}_{23}&\vec{B}_{23}\\\vec{A}_{31}&\vec{B}_{31}\\\vec{A}_{12}&\vec{B}_{12}}}
\end{align}
Hence
\begin{align}
 \mydet{\vec{A}_{23}&\vec{B}_{23}}&=\mydet{2&2\\2&-2}=\myvec{-4-4}=-8\\
 \mydet{\vec{A}_{31}&\vec{B}_{31}}&=\mydet{2&3\\-2&1}=\myvec{2-(-6)}=8\\
 \mydet{\vec{A}_{12}&\vec{B}_{12}}&=\mydet{3&2\\1&2}=\myvec{6-2}=4
\end{align}
which can be represented in matrix form as
\begin{align}
\text{perpendicular to vector represented by}
\hat{\vec{c}} &=\frac{\overrightarrow{\vec{c}}}{\abs{\overrightarrow{\vec{c}}}} \label{eq:6}\\
\overrightarrow{\vec{c}}=\vec{A} \times \vec{B}&=\myvec{8\\-8\\-4} 
\end{align}
Hence
\begin{align}
\abs{\overrightarrow{\vec{c}}}&={\sqrt{8^2+(-8^2)+(-4^2)}}=12  
\end{align}
Here substiuing the values in \eqref{eq:6} so we get
\begin{align}
\hat{\vec{c}}=\myvec{1\\12}\myvec{8\\-8\\-4}
\end{align}
\begin{align}
\hat{\vec{c}}=\myvec{1\\3}\myvec{2\\-2\\-1}
\end{align}
\end{enumerate}
\end{document}