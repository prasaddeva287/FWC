\documentclass[12pt]{article}
\usepackage{graphicx}
\usepackage[none]{hyphenat}
\usepackage{graphicx}
\usepackage{listings}
\usepackage[english]{babel}
\usepackage{graphicx}
\usepackage{caption} 
\usepackage{booktabs}
\usepackage{array}
\usepackage{amssymb} % for \because
\usepackage{amsmath}   % for having text in math mode
\usepackage{extarrows} % for Row operations arrows
\usepackage{listings}
\usepackage[utf8]{inputenc}
\lstset{
  frame=single,
  breaklines=true
}
\usepackage{hyperref}
  
%Following 2 lines were added to remove the blank page at the beginning
\usepackage{atbegshi}% http://ctan.org/pkg/atbegshi
\AtBeginDocument{\AtBeginShipoutNext{\AtBeginShipoutDiscard}}


%New macro definitions
\newcommand{\mydet}[1]{\ensuremath{\begin{vmatrix}#1\end{vmatrix}}}
\providecommand{\brak}[1]{\ensuremath{\left(#1\right)}}
\newcommand{\solution}{\noindent \textbf{Solution: }}
\newcommand{\myvec}[1]{\ensuremath{\begin{pmatrix}#1\end{pmatrix}}}
\providecommand{\norm}[1]{\left\lVert#1\right\rVert}
\providecommand{\abs}[1]{\left\vert#1\right\vert}
\let\vec\mathbf

\begin{document}

\begin{center}
\title{\textbf{LINE}}
\date{\vspace{-5ex}} %Not to print date automatically
\maketitle
\end{center}

\section{11$^{th}$ Maths - EXERCISE-10.3}
\begin{enumerate}
\item The line through the points (h, 3) and (4, 1) intersects the line 7x- 9y- 19= 0 at right angle. Find the value of h.
\end{enumerate}
\section{SOLUTION}
Given points are P=\myvec{h\\ 3},Q=\myvec{4\\ 1},B=\myvec{9\\ 7}
\begin{align}
\vec{A}=\vec{Q}-\vec{P}&=\myvec{4\\ 1}-\myvec{h\\ 3}\\
\vec{A}&=\myvec{4-h\\ -2}
\end{align}
The formula for the line 
\begin{align}
\theta&=90^\circ\\
cos\theta&=90^\circ=0\\
cos\theta&=\frac{\vec{A}^\top\vec{B}}{\norm{A}\norm{B}}\\
&=\myvec{4-h& -2}\myvec{9\\ 7}\\
&=9\myvec{4-h}-14\\
22&=9h\\
h&=\frac{22}{9}
\end{align}
\section{Figure}
\begin{figure}[h]
\centering
\includegraphics[width=\columnwidth]{fig.pdf}
\caption{line}
		\label{fig:Figure}
\end{figure}
\end{document}