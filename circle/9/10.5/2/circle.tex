\documentclass[12pt]{article}
\usepackage{graphicx}
\usepackage{amsmath}
\usepackage{mathtools}
\usepackage{gensymb}
 \usepackage[latin1]{inputenc}
       \usepackage{fullpage}
       \usepackage{color}
       \usepackage{array}
       \usepackage{longtable}
       \usepackage{calc}
       \usepackage{multirow}
       \usepackage{hhline}
       \usepackage{ifthen}
\usepackage{graphicx}
\def\inputGnumericTable{}

\newcommand{\mydet}[1]{\ensuremath{\begin{vmatrix}#1\end{vmatrix}}}
\providecommand{\brak}[1]{\ensuremath{\left(#1\right)}}
\providecommand{\norm}[1]{\left\lVert#1\right\rVert}
\newcommand{\solution}{\noindent \textbf{Solution: }}
\newcommand{\myvec}[1]{\ensuremath{\begin{pmatrix}#1\end{pmatrix}}}
\let\vec\mathbf

\begin{document}
\begin{center}
\textbf\large{CHAPTER-9 \\ CIRCLES}

\end{center}
\begin{enumerate}
\section{EXERCISE-10.5}
\item A chord of a circle is equal to the radius of the circle. Find the angle subtended by the chord at a point on the minor arc and also at a point on the major arc.
\section{SOLUTION}
The input parameters are the length\\
\begin{table}[ht!]
	\input{/home/prasaddeva287/prasad/circle/9-10.5-2/table/table.tex}
\caption{chords intersecting in a circle}
\label{table}	
\end{table}
\\
Take three points Q,R and P on a unit circle  at angles $\theta,\alpha,\text{ and }\beta$.Then
\begin{align}
	\vec{Q} = \myvec{cos\theta\\sin\theta},
	\vec{R} = \myvec{cos\alpha\\sin\alpha},
	\vec{S} = \myvec{cos\beta\\sin\beta}
\end{align}
\begin{align}
	\cos\angle QRP&= \frac{(Q-R)(P-R)}{\mydet{Q-R}\mydet{P-R}}\label{2}
\end{align}
Where
\begin{align}
(Q-R, P-R)&= (\cos\theta-\cos\alpha,\sin\theta-\sin\alpha),\myvec{1-\cos\alpha,& o-\sin\alpha}\\
&=(\cos\theta-\cos\alpha)\cos\alpha+(\sin\theta-\sin\alpha)\\
&=2\sin\frac{\theta-\alpha}{2}\sin\frac{\theta+\alpha}{2}\cos\alpha+2\cos\frac{\theta+\alpha}{2}\sin\frac{\theta-\alpha}{2}\\
&=(\cos\alpha-\cos\theta)\cos\alpha+(\sin\theta-\sin\alpha)\label{6}
\end{align}
\begin{align}
\mydet{Q-R}^2\mydet{P-R}^2 &= ((\cos\theta-\cos\alpha)^2+(\sin\theta-\sin\alpha)^2)
	((1-\cos\alpha)^2+(0-\sin\alpha)^2)\\
	&=(2-2\cos\theta\cos\alpha-2\sin\theta\sin\alpha)(2-\cos\alpha)\label{8}
\end{align}
substituing the \eqref{6} and \eqref{8} in \eqref{2}
\begin{align}
\cos\angle QRP&=\frac{2.079}{4.323}\\
\angle QRP&=\cos^{-1}0.480\\
\angle QRP&=66\degree
\end{align}
\begin{align}
\cos\angle QSP&= \frac{ (Q-S) (P-S)}{\mydet{Q-S}\mydet{P-S}}\label{12}
\end{align}
\begin{align}
(Q-S, P-S) &= (\cos\theta-\cos\beta,\sin\theta-\sin\beta),\myvec{1-\cos\beta,& o-\sin\beta}\\
&=(\cos\theta-\cos\beta)\cos\beta+(\sin\theta-\sin\beta)\\
&=2\sin\frac{\theta-\beta}{2}\sin\frac{\theta+\beta}{2}\cos\beta+2\cos\frac{\theta+\beta}{2}\sin\frac{\theta-\beta}{2}\\
&=(\cos\beta-\cos\theta)\cos\beta+(\sin\theta-\sin\beta)\label{16}
\end{align}
\begin{align}
\mydet{Q-S}^2\mydet{P-S}^2 &= ((\cos\theta-\cos\beta)^2+(\sin\theta-\sin\beta)^2)
	((1-\cos\beta)^2+(0-\sin\beta)^2)\\
	&=(2-2\cos\theta\cos\beta-2\sin\theta\sin\beta)(2-\cos\beta)\label{18}
\end{align}
substituing the \eqref{16} and \eqref{18} in \eqref{12}
\begin{align}
\cos\angle QSP&=\frac{1.048}{1.098}\\
\angle QSP&=\cos^{-1}0.954\\
\angle QSP&=17\degree
\end{align}
\section{FIGURE}
\begin{figure}[h]
\centering
\includegraphics[width=\columnwidth]{circle.png}
\caption{circle}
		\label{fig:Figure}
\end{figure}
\end{enumerate}
\end{document}